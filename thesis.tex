% -*-coding: utf-8-*-
% This is an AMS-LaTeX v. 1.2 File.

\documentclass{report}

%\usepackage{pscyr}
%\renewcommand{\rmdefault}{fjn}
%\renewcommand{\ttdefault}{fcr}

%\usepackage{showkeys}
\usepackage[T2A]{fontenc}
\usepackage[utf8x]{inputenc}
\usepackage[english,russian]{babel}
\usepackage{expdlist}
\usepackage[pdftex]{graphicx}
\usepackage{amsmath}
\usepackage{amssymb}
\usepackage{amsthm}
\usepackage{amsfonts}
\usepackage{amsxtra}
\usepackage{sty/dbl12}
\usepackage{srcltx}
\usepackage{epsfig}
\usepackage{verbatim}
\usepackage{sty/rac}
%\usepackage[russian]{sty/ralg}
\usepackage{listings}
\usepackage{placeins}
%\usepackage{caption}
%\usepackage{floatrow}
\usepackage{caption}
\usepackage{dcolumn}

\captionsetup[table]{position=t,justification=raggedright,slc=off}

%\usepackage[
%    top    = 2.00cm,
%    bottom = 2.00cm,
%    left   = 3.00cm,
%    right  = 1.50cm]{geometry}
\hoffset = -10mm
\voffset = -20mm
\textheight = 230mm
\textwidth = 165mm

%%%%%%%%%%%%%%%%%%%%%%%%%%%%%%%%%%%%%%%%%%%%%%%%%%%%%%%%%%%%%%%%%%%%%%%%%%%%%%

% Redefine margins and other page formatting

%\setlength{\oddsidemargin}{0.5in}

% Various theorem environments. All of the following have the same numbering
% system as theorem.

\theoremstyle{plain}
\newtheorem{theorem}{Теорема}
\newtheorem{prop}[theorem]{Утверждение}
\newtheorem{corollary}[theorem]{Следствие}
\newtheorem{lemma}[theorem]{Лемма}
\newtheorem{question}[theorem]{Вопрос}
\newtheorem{conjecture}[theorem]{Гипотеза}
\newtheorem{assumption}[theorem]{Предположение}

\theoremstyle{definition}
\newtheorem{definition}[theorem]{Определение}
\newtheorem{notation}[theorem]{Обозначение}
\newtheorem{condition}[theorem]{Условие}
\newtheorem{example}[theorem]{Пример}
\newtheorem{algorithm}[theorem]{Алгоритм}
%\newtheorem{introduction}[theorem]{Introduction}

\renewcommand{\proof}{\\\textbf{Доказательство.}~}

%\def\startprog{\begin{lstlisting}[language=Java,basicstyle=\normalsize\ttfamily]}

%\theoremstyle{remark}
%\newtheorem{remark}[theorem]{Remark}
%\include{header}
%%%%%%%%%%%%%%%%%%%%%%%%%%%%%%%%%%%%%%%%%%%%%%%%%%%%%%%%%%%%%%%%%%%%%%%%%%%%%%%

\numberwithin{theorem}{chapter}        % Numbers theorems "x.y" where x
                                        % is the section number, y is the
                                        % theorem number

%\renewcommand{\thetheorem}{\arabic{chapter}.\arabic{theorem}}

%\makeatletter                          % This sequence of commands will
%\let\c@equation\c@theorem              % incorporate equation numbering
%\makeatother                           % into the theorem numbering scheme

%\renewcommand{\theenumi}{(\roman{enumi})}

%%%%%%%%%%%%%%%%%%%%%%%%%%%%%%%%%%%%%%%%%%%%%%%%%%%%%%%%%%%%%%%%%%%%%%%%%%%%%%


%%%%%%%%%%%%%%%%%%%%%%%%%%%%%%%%%%%%%%%%%%%%%%%%%%%%%%%%%%%%%%%%%%%%%%%%%%%%%%%

%This command creates a box marked ``To Do'' around text.
%To use type \todo{  insert text here  }.

\newcommand{\todo}[1]{\vspace{5 mm}\par \noindent
\marginpar{\textsc{ToDo}}
\framebox{\begin{minipage}[c]{0.95 \textwidth}
\tt #1 \end{minipage}}\vspace{5 mm}\par}

%%%%%%%%%%%%%%%%%%%%%%%%%%%%%%%%%%%%%%%%%%%%%%%%%%%%%%%%%%%%%%%%%%%%%%%%%%%%%%%

\binoppenalty=10000
\relpenalty=10000

\begin{document}


% Begin the front matter as required by Rackham dissertation guidelines

\initializefrontsections

\pagestyle{title}

\begin{center}
Санкт-Петербургский национальный исследовательский университет \\ информационных технологий, механики и оптики

\vspace{2cm}

Кафедра компьютерных технологий

\vspace{3cm}

{\Large А. К. Касс}

\vspace{2cm}

\vbox{\LARGE\bfseries
Определение района поиска\\с учетом топографических\\
свойств местности}

\vspace{4cm}

Магистерская диссертация

\vspace{1cm}

{\Large Научный руководитель: С. Ю. Жуков}

\vspace{5cm}

Санкт-Петербург\\ 2014
\end{center}

\newpage

\setcounter{page}{2}
\pagestyle{plain}

%\dedicationpage{Put a dedication here}
% Dedication page

%\startacknowledgementspage
% Acknowledgements page
%{Put Acknowledgements here}

% Table of contents, list of figures, etc.
\tableofcontents
%\listoffigures


\def\t#1{\mbox{\texttt{\hbox{#1}}}}
\def\b#1{\textbf{#1}}
\def\tb#1{\t{\b{#1}}}

\def\cln#1{\t{#1}}
\def\pcn#1{\t{#1}}
\newcommand{\p}{\par Здесь будет текст...}

\def\drawfigure#1#2#3{
        \begin{figure}[ht]
        \centerline{ \includegraphics[width=8cm]{img/#1}}
        \caption{#2}
        \label{#3}
        \end{figure}
}
\def\drawfigurex#1#2#3#4{
        \begin{figure}[ht]
        \centerline{ \includegraphics[#4]{img/#1}}
        \caption{#2}
        \label{#3}
        \end{figure}
}

% Chapters
\startthechapters
% -*-coding: utf-8-*-
\startprefacepage

При при проведении поисково-спасательных операций, одним
из основных факторов, определяющих ее успех, является время.
Поэтому правильное определение района поиска может стать залогом
успеха операции.

\begin{description}
\item[Исходный район] -- область, где в последний раз был замечен объект поиска.
\item[Достижимый район на время $t$] -- множество точек, достижимых
из точек исходного района за время $t$.
\item[Район поиска] -- область в которой осуществляется поиск.
\item[Корректный район поиска на время $t$] -- район поиска, полностью
содержащий в себе достижимый район на время $t$
\end{description}

Задача заключается в определении корректного района поиска на заданный
момент времени, по известным картографическим данным, модели
объекта поиска и исходному району, отличающийся от достижимого района
не более чем на заданную величину.

\drawfigure{example}{Пример района поиска}{img:ex}

На сегодняшний, при планировании поиска и спасания производится
очень грубая оценка района поиска, что ведет к бессмысленной трате
времени и ресурсов на поиск в местах, где объкта поиска точно нет \cite{SAR}.

Очевидно, что для этой задачи есть необходимость получать
районы поиска на разные моменты времени при одних и тех же
входных данных. Например просматривать динамику расширения района
поиска с течением времени. Это означает что запросы
районов поиска будут массовыми.

Массовая задача в \cite{PrSh} определена следующим образом. Существует
фиксированный набор входных данных $S$. Требуется вычислить массовый
запрос $Q$, то есть ответить на некоторый поставленный вопрос для каждого
запроса из $Q$. Иногда такие задачи решаются в два этапа: предобработка
(pre-processing) и вычисление запросов на некоторой структуре данных,
формирующейся на этапе предобработки и ускоряющих вычисления, что позволяет
сократить суммарное время по сравнению с последовательным решением
исходной задачи для каждого запроса.

В данной работе описан новый метод массового построения районов
поиска на заданные моменты времени, которые с одной стороны являются корректными,
а с другой стороны отличаются от достижимых на эти моменты
времени районы не более чем на заданную величину.

\FloatBarrier

%-*-coding: utf-8-*-
\chapter{Существующие методы}

Задача определения района поиска сводится к задаче
построения карты расстояний до некоторого исходного множества точек
в условии наличия полигональных препятствий разного веса
(Single-Source Weighted Region Problem, SSWRP). Эта задача алгебраически
неразрешима в виду неразрешимости задачи нахождения кратчайшего пути в
тех же условиях \cite{WRPU}. Поэтому для ее решения используются методики
аппроксимации и оптимизации пути численными методами.

\section{Дискретизация пространства}

Одним из самых распространенных способов решения задач подобного рода является
дискретизация пространства с последующим построением графа из элементов
подазбиения \cite{GRID1, GRID2, GRID3}.
В основном в таких методах пространство подразбивается регулярной сеткой
(рис. \ref{img:grid}), реже адаптивно (рис. \ref{img:qt}), структурой схожей
с квадродеревом. Точность приближения длины путей
в таких графах на прямую связана характерными размерами элементов подразбиения,
и количеством соседних ячеек, с которыми соединена каждая.
В результате данные методы используют неоправданно большой объем памяти, при
сравнительно малой точности, хотя они и просты в реализации.

\drawfigurex{gridexample}{Подразбиение карты по сетке}{img:grid}{width=10cm}
\drawfigurex{qtexample}{Адаптивное подразбиение карты}{img:qt}{width=10cm}

\FloatBarrier

\section{Точки Штейнера}
Данный метод базируется на триангуляции исходных данных \cite{STAINER1, STAINER2}.
Его работа основана  на двух наблюдениях. Во-первых, в триангуляции исходных
данных каждый треугольник имеет постоянный вес, в результате чего кратчайший
путь внутри треугольника -- прямая. Во-вторых, кратчайший путь подчиняется
закону Снелля из оптики \cite{SNELL}.

Закон Снелля позволяет по углу падения луча на границу раздела двух сред и
скоростям его распространения в этих средах получить угол преломления --
угол, под которым произойдет выход луча из границы раздела сред (рис. \ref{img:snelllaw}).
В виду того, что свет движется по кратчайшему пути (принцип Ферма), то этот закон
имеет прямую аналогию в планировании маршрутов.
Рассмотрим ребро триангуляции лежащее на границе районов с разной проходимостью.
Пусть $\theta_1$ -- угол, под которым кратчайший путь входит в это ребро,
$theta_2$ -- угол, под которым он выходит, $v_1, v_2$ -- скорости движения
в районах до и после ребра соответственно. Тогда имеет место равенство:
\begin{equation}
\sin\theta_2 = \sin\theta_1\frac{v_2}{v_1}
\end{equation}

\drawfigure{snelllaw}{Преломление по закону Снелля}{img:snelllaw}

Угол $\theta_c = \arcsin\frac{v_2}{v_1}$ называется критическим углом.
Возможны эффекты так называемого критического отражения от ребра
(рис. \ref{img:crref}) и критического использования ребра
(рис. \ref{img:crusage}) \cite{SNELL}. От ребра происходит критической
отражения маршрута, когда кратчайший путь между двумя точками района
с одной проходимостью выходит из него под критическим углом, а затем
входит обратно, тоже под критическим углом. Это происходит в случае
сильно большей скорости движения в соседнем районе. Критическое
использование ребра происходит в случае прохождения пути между двух районов,
между которыми есть линейный объект с более низкой чем в районах проходимостью.
В этом случае путь под критическим углом входит в это ребро, идет некоторое
расстояние по нему, а затем под критическим углом выходит во второй район.

\drawfigurex{critreflection}{Критическое отражение от ребра}{img:crref}{width=12cm}
\drawfigurex{critusage}{Критическое использование ребра}{img:crusage}{width=12cm}

Закон Снелля позволяет создать более умную схему подразбиения, по которой
на ребра триангуляции, на которых происходит изменение проходимости районов,
добавляются дополнительные точки -- точки Штейнера (рис. \ref{img:stainer}).
Далее триангуляция используется как граф, в котором внутри треугольника точки
Штейнера и вершины образуют полный граф.

\drawfigure{stainerpoints}{Пример расположения точек Штейнера}{img:stainer}

Точность такого метода связана с количеством добавленных точек Штейнера, а
также их распределением по ребру. Во многих работах советуется располагать
больше точек у вершин \cite{STAINER1, STAINER2}.

\FloatBarrier

\section{Continuous Dijkstra}

\FloatBarrier

\section{Pathnet}

Алгоритм основан на построении графа, который бы с заданной точностью приближал
расстояния на плоскости \cite{PATHNET}.

Из каждой вершины пускаются $k$ лучей, которые далее трассируются в соответствии
с законом Снелля. Пара соседних лучей образует угол. Трассировка идет до тех пор,
пока все лучи, лежащие внутри этого угла ведут себя одинаково. А именно, пока не
произойдет одно из двух событий:
\begin{itemize}
\item \textbf{Лучи-границы конуса прошли через разные
ребра одной грани триангуляции (рис. \ref{img:pnvert})}\\
В этом случае добавляется ребро между исходной вершиной и вершиной, разделившей
эти два луча. Между этими двумя вершинами кратчайший путь обязательно лежит в
конусе. Производится поиск кратчайшего пути, например бинарным поиском, и он
запоминается в ребре, весом ребра назначается вес кратчайшего пути.
\item \textbf{Один из лучей-границ вошел в ребро под углом,
большим критического (рис. \ref{img:pncrit})} \\
В этом случае в граф добавляется эта точка -- критическая вершина.
Она соединяется с другими критическими вершинами этого ребра, а также его концами.
Также как и в предыдущем случае, строится кратчайший путь до исходной вершины,
входящий в критическую вершину под критическим углом.
\end{itemize}

\drawfigure{pathnetvertex}{Остановка трассировки из-за вершины}{img:pnvert}
\drawfigure{pathnetcrit}{Остановка трассировки при пересечении под критическим углом}{img:pncrit}

Поиск кратчайшего пути осуществляется вставкой начала и конца пути в граф, а затем
поиском пути с помощью алгоритма дейкстры.

Время работы этого алгоритма $O(kn^3)$, точность $O(\frac{W/\omega}{k\theta_{min}})$,
где $W$ -- максимальный конечный вес, $\omega$ -- минимальный ненулевой вес,
$\theta_{min}$ -- минимальный угол триангуляции.

\FloatBarrier

%-*-coding: utf-8-*-
\everymath{\displaystyle}

\chapter{Решение задачи определения района поиска предложенным методом}
\section{Нахождение функции кратчайшего расстояния}
\subsection{Обработка исходных картографических данных}
Картографические данные поступают на вход алгоритма в
виде ломанных и многоугольников, для каждой из которых указан тип местности
для этого объекта (лес, дорога, озеро...). По этим данным строится триангуляция
Делоне (Constrained Delaunay Triangulation). Ее ребрам и треугольникам
соответствуют атрибуты -- множество типов местности через которые они проходят.

\drawfigurex{examplemap}{Пример картографических данных}{img:exmap}{width=10cm}
\drawfigurex{exampletriangulation}{Пример триангулированных}{img:extri}{width=10cm}

Эта триангуляция обладает тем свойством, что все внутренние точки каждого ее
треугольника имеют одинаковый тип местности, откуда следует, что внутри
треугольников кратчайшие расстояние измеряется по прямой. То же верно и
для ребер триангуляции.

Стоит обратить внимание на то, что тип местности для ребра треугольника
не обязательно совпадает с типом местности самого треугольника. Достаточно
рассмотреть что произойдет с дорогой проходящей через лес. Треугольники
инцидентные дороге будут иметь тип местности "лес", тогда как ребра
порожденные дорогой будут иметь тип местности "дорога".

\FloatBarrier

\subsection{Сужение функции кратчайшего расстояния}
Функция кратчайшего расстояния $d(P)$ определена на плоскости, но
задание такой сложной функции на плоскости не тривиальная задача.
Поэтому удобно будет рассматривать эту функцию на линейных объектах.
В чем нам поможет следующее утверждение.

{\prop\label{sh_path}
$\forall T: T$ -- не содержит точек исходного района,
$\forall P \in T, d(P) = \min_{Q \in \delta T} \{d(Q) + dist(Q, P)\}$}
\begin{proof}
Рассмотрим кратчайший путь в $P$. Так как путь начинается вне $T$,
а заканчивается внутри, то существует $Q \in \delta T$ -- точка
где кратчайший путь пересекает границу $T$.

Покажем, что в этой точке достигается минимум. предположим обратное.
Рассмотрим другую точку $R \in \delta T$, в которой достигается минимум.
Получаем $d(P) = d(Q) + dist(Q, P) > d(R) + dist(R, P) = d(P)$ -- противоречие.
\end{proof}

\drawfigure{proofex}{Кратчайший путь до внутренней точки треугольника}{img:proofex}

В итоге получается, что кратчайшее расстояние до внутренних точек треугольника,
полностью определяется кратчайшим расстоянием до его границ.

То есть задача нахождения кратчайшего расстояния до всех точек плоскости
может быть разбита на две части:
\begin{enumerate}
\item Нахождение кратчайших расстояний до границ некоторого подразбиения плоскости
\item Нахождение кратчайших расстояний внутри областей подразбиения, как
$d(P) = \min_{Q \in \delta T} \{d(Q) + dist(Q, P)\}$
\end{enumerate}

Теперь рассмотрим триангуляцию, получившуюся в результате обработки исходных данных.
Далее используем функцию $lerp(S, t) = tS_0 + (1-t)S_1$ -- интерполяция точки отрезка.
Назовем $d_S(t) = d(lerp(S, t))$ сужением функции $d(P)$ на отрезок $S$.
Если найти кратчайшие расстояния до ее ребер, то кратчайшее расстояние до
его внутренних точек будет
$d(P) = \min_{S \in \{S^1, S^2, S^3\} } \min_{t \in [0; 1] } \{d(lerp(S, t)) + len(lerp(S, t), P)\}$,
где $len(P, Q)$ -- длина отрезка $PQ$, а $S^1, S^2, S^3$ -- стороны треугольника.

\FloatBarrier

\subsection{Структуры данных}
Для возможности учета линейных объектов ребра триангуляции порождают
два типа объектов (далее просто ребра):
\begin{itemize}
\item Ребра треугольника -- ребра граней триангуляции.
Тип местности ребра соответствует типу местности грани.
\item Промежуточные ребра -- ребра триангуляции.
Тип местности соответствует типу местности ребра триангуляции.
\end{itemize}
Каждое ребро триангуляции, кроме находящихся на границе, порождает
два ребра треугольника (по одному на каждый смежный треугольник)
и одно промежуточное ребро.

\drawfigure{structure}{Пример структуры (треугольники смещены для удобства отображения)}{img:struct}

Триангуляция хранится в реберном списке с двойной связностью
(РСДС, DCEL) -- структуре данных позволяющей удобно оперировать
планарным подразбиением плоскости.

РСДС -- структура для хранения плоского прямолинейного графа (ППЛГ, PSLG) \cite{PrSh}.
Состоит из ориентированных ребер, реберных узлов и граней. Ребра ориентированы в сторону
обхода граней против часовой стрелки. Каждое ребро
указывает на следующее ребро грани, предыдущие ребро грани, смежное противоположно
направленное ребро грани, лежащей справа от ребра, а также реберный узел начала
ребра и грань, лежащая слева от ребра. Реберный хранит указатель на исходящее
из него ребро. Грань хранит одно из своих ребер.

\drawfigurex{dcel}{Пример РСДС}{img:dcel}{width=10cm}

Каждому ребру присваивается уникальный индекс, который далее
будет использоваться для получения его атрибутов.
Эти индексы хранятся в атрибутах РСДС:
\begin{itemize}
\item В атрибутах грани в виде троек $\langle I^1, I^2, I^3 \rangle$ --
индексы ребер треугольника.
\item В атрибутах ребер  в виде пар $\langle I, D \rangle$, где $I$ --
индекс промежуточного ребра, а $D$ -- показывает сонаправлено или
противонаправлено промежуточное ребро с ребром РСДС. Второй
параметр необходим, так как одному ребру триангуляции соответствуют
два противонаправленных ребра РСДС. Направление промежуточного пебра
выбирается произвольно.
\end{itemize}

Каждому ребру соответствуют атрибуты:
\begin{itemize}
\item тип местности ребра;
\item текущее приближение $d_S(t)$.
\end{itemize}

Чтобы далее последовательно приближать $d_S(t)$, нужно ввести связность между
ребрами получившейся структуры.
\begin{itemize}
\item С ребра треугольника можно перейти на:
  \begin{itemize}
  \item другое ребро этого же треугольника;
  \item смежное ему промежуточное ребро.
  \end{itemize}
\item С промежуточного ребра можно перейти на:
  \begin{itemize}
  \item промежуточные ребра, инцидентные его началу и концу;
  \item смежное ему ребро треугольника.
  \end{itemize}
\end{itemize}

\drawfigurex{intermdist}{Распространение с промежуточного ребра}{img:intermdist}{width=12cm}
\drawfigurex{tridist}{Распространение с ребра треугольника}{img:tridist}{width=12cm}

Такая связность позволяет учитывать множество аспектов движения
объекта поиска. Например ему можно запретить пересекать реки,
можно разрешить, но ввести штраф, равный времени пересечения реки.
Для этого вводятся две величины:
\begin{description}
\item[Коэффициент проходимости] величина характеризующая сложность прохождения
одного метра. В нашем случае это $\frac{1}{v}$, где $v$ -- скорость движения
по данному типу местности. Задается для ребра.
\item[Штраф] величина, добавляющаяся к функции кратчайшего расстояния при
переходе с одного ребра на другое. Определяется по типу местности ребер,
между которыми осуществляется переход. Для случая перехода с промежуточного
ребра на промежуточное, влияют все ребра инцидентные вершине, через которую
осуществляется переход. Последнее позволяет проверить, пересекается ли при
этом река, пересекает ли реку дорога (свидетельство наличия моста).
\end{description}


\FloatBarrier

\subsection{Алгоритм}
Суть алгоритма заключается в последовательном определении сужения функции
кратчайшего расстояния для каждого ребра триангуляции.

Назовем $\displaystyle d'_S(t) = \min_{t' \in [0; 1]}\{d_{S'}(t') + dist(lerp(S', t'), lerp(S, t))\}$
обновлением функции $d_S(t)$ ребра $S$ от ребра $S'$. Эта функция
соответствует длинам текущего приближения кратчайших путей до ребра $S$,
проходящих через ребро $S'$.

В приоритетной очереди находятся события. Событие -- это пара
$\langle S, d'_S(t) \rangle$. Приоритет события -- $\displaystyle\min_{t \in [0; 1]}\{d'_S(t)\}$.

\begin{description}
\item[Инициализация]\hfill \\
\begin{enumerate}
\item Для каждого ребра $S$, $d_S(t) \gets \infty$.
\item Для каждого ребра $S: S$ -- внутри исходного района, $d_S(t) \gets 0$.
\item Для каждой вершины $V$ исходного района, для каждого инцидентного ей
ребра $S$:
  \begin{enumerate}
  \item $d'_S(t) \gets len(S)wt'$, где $len(s)$ -- длина ребра $S$,
    $w$ -- его коэффициент проходимости, $t' = t$ или $1 - t$, в зависимости от
    того $V$ начало или конец $S$ соответственно.
  \item Добавить событие $\langle S, d'_S(t) \rangle$ в очередь.
  \end{enumerate}
\end{enumerate}

\item[Шаг алгоритма]\hfill \\
\begin{enumerate}
\item Из очереди извлекается событие $\langle S, d'_S(t) \rangle$ с наименьшим приоритетом.
\item Если $\forall x \in [0; 1]: d_S(t) \leq d'_S(t)$, то событие игнорируется
\item $d_S \gets lower\_envelope(d_S, d'_S)$, где $lower\_envelope$ --
нижняя огибающая, поточечный минимум функций.
\item Для каждого ребра $S'$, связанного с $S$:
  \begin{enumerate}
  \item Строится $d'_{S'}(t)$ -- обновление для ребра $S'$ от ребра $S$.
  \item В очередь добавляется событие $\langle S', d'_{S'}(t) \rangle$
  \end{enumerate}
\end{enumerate}

\item[Конец алгоритма]\hfill \\
В очереди не осталось событий.
\end{description}

\FloatBarrier

\subsection{Упрощение функции кратчайшего расстояния}
Рассмотрим подробнее что происходит с функцией кратчайшего расстояния
при ее обновлении. Для это детально рассмотрим процесс формирования
функции обновления. Пусть $d'_{S'}(t)$ -- обновление для ребра $S'$ от ребра
$S$ с функцией кратчайшего расстояния $d_s(t)$. Рассмотрим случаи.
\\
\\
\textbf{$S$ -- ребро треугольника, $S'$ -- промежуточное ребро, или наоборот}\\
Так как $S$ и $S'$ совпадают, то:
\begin{equation} \label{eq:itd}
d'_{S'}(t) = d_S(t') + penalty(S, S')
\end{equation}
где $t' = t$ или $1 - t$ в зависимости от взаимной ориентации $S$ и $S'$,\\
$penalty(S, S')$ -- штраф за переход с ребра $S$ на $S'$.
\\
\\
\textbf{$S$ и $S'$ -- промежуточные ребра}\\
Так как $S$ и $S'$ имеют общую точку:
\begin{equation} \label{eq:iid}
d'_{S'}(t) = d_S(t_1) + len(S')wt' + penalty(S, S')
\end{equation}
где $t_1 = 0$ или $1$ в зависимости от того, какому концу $S$
инцидентно ребро  $S'$,\\
$t' = t$ или $1 - t$ в зависимости от того, какому концу $S'$
инцидентно ребро $S$,\\
$penalty(S, S')$ -- штраф за переход с ребра $S$ на $S'$,\\
$w$ -- коэффициент проходимости $S'$.
\\
\\
\textbf{$S$ и $S'$ -- ребра треугольника}\\
Так как расстояния внутри треугольника измеряются по прямой:
\begin{equation} \label{eq:ttd}
d'_{S'}(t) = \min_{t' \in [0; 1]}\{d_S(t') + w|lerp(S', t') - lerp(S, t)|\}
\end{equation}
где $w$ -- коэффициент проходимости внутри треугольника.\\
Обозначим $D(t, t') = d_S(t') + w|lerp(S', t) - lerp(S, t')|$,
Рассмотрим уравнение:
\begin{equation} \label{eq:ttder}
\frac{\partial D(t, t')}{\partial t'} = 0
\end{equation}
Пусть $T^*$ -- множество его корней (зависят от $t$), $T^{**}$ -- множество, на котором
функция $D(t, t')$ не имеет производной. Тогда $d'_{S'}(t)$ выражается явно:
\begin{equation} \label{eq:ttd2}
d'_{S'}(t) = \mathop{lower\_envelope}
  \{D(t, t^*(t))\}^{t^*(t) \in (T^* \cup T^{**}) \cup \{0; 1\} \& t^*(t) \in [0;1] }
\end{equation}

Заметим, что в этих формулах не учитывается передвижение
по обновляемому ребру, поэтому $d'_S(t)$ необходимо релаксировать, а
именно $d'_S(t) \gets \mathop{lower\_envelope}{M_S(d'_S(t_0), t - t_0)}_{t_0 \in [0; 1]}$,
где $M_S(d, t) = d + w_S\mathop{len}(S)|t|$ -- функция движения по ребру,
показывающая распределение расстояний по ребру от точки, до которой
расстояние $d$ (рис. \ref{img:relax}).

\drawfigurex{relax}{Релаксация обновления}{img:relax}{width=10cm}

Из формул \ref{eq:itd} и \ref{eq:iid} видно, что при первых двух типах
обновления $d_S(t)$ не меняет свою сложность. Тогда как при третьем
типе обновления ее сложность увеличивается. Поэтому возникает
необходимость отказа от точного решения. Можно зафиксировать
$d_S(t)$ в каком либо виде (кусочно-линейная, кусочно квадратичная),
в котором будет возможно строить нижнюю огибающую и находить обновления
третьего вида. Когда в результате обновления третьего вида $d'_S(t)$
получается сложнее чем нужно, ее необходимо аппроксимировать функцией
$d''_S(t)$ зафиксированного нами вида (рис. \ref{img:approxex}).
Для сохранения корректности района
необходимо выполнение условия $\forall t \in [0; 1]: d''_S(t) \leq d'_S(t)$.

\drawfigure{approxex}{Пример приближения линейной функцией}{img:approxex}

Рассмотрим использование кусочно линейных функций подробнее.
Пусть $d_S(t) = at + b$, для $t \in [t_0; t_1]$, тогда
$D(t, t') = at' + b + w|lerp(S', t) - lerp(S, t')|$.
Рассмотрим случай, когда начало $S$ совпадает с началом $S'$ в точке $P$,
остальные случаи получаются заменой переменных.
Пусть $k = S_1 - P$, $k' = S'_1 - P$, тогда
\[
\begin{aligned}
D(t, t') &= at' + b + w|P + tk' - (P + tk)| \\
&= at' + b + w|tk' - t'k| \\
&= at' + b + w\sqrt{t^2k'^2 - 2tt'(k', k) + t'^2k^2} \\
\\
\frac{\partial D(t, t')}{\partial t'} &= a + w\frac{-t(k', k) + t'k^2}
{\sqrt{t^2k'^2 - 2tt'(k', k) + t'^2k^2}} = 0\\
\\
a^2(t^2k'^2& - 2tt'(k', k) + t'^2k^2) = w^2(t^2(k', k)^2 - 2tt'(k', k)k^2 + t'^2k^4) \\
t'^2(a^2k^2& - w^2k^4) - 2tt'(a^2(k', k) - w^2k^2(k', k)) + t^2(a^2k'^2 - w^2(k', k)^2) = 0\\
\\
\frac{D}{4} &= (a^2 - w^2k^2)^2(k', k)^2 - k^2(a^2 - w^2k^2)(a^2k'^2 - w^2(k', k)^2)\\
&= (a^2 - w^2k^2)(a^2(k', k)^2 - w^2k^2(k', k)^2 - a^2k'^2k^2 + w^2(k', k)^2k^2)\\
&= a^2(w^2k^2 - a^2)(k'^2k^2 - (k', k)^2)\\
\frac{t'}{t} &= \frac{(k', k)(a^2 - w^2k^2) \pm \sqrt{a^2(w^2k^2 - a^2)(k'^2k^2 - (k', k)^2)}}
{k^2(a^2 - w^2k^2)}\\
&= \frac{(k', k) \pm \sqrt{\frac{a^2(k'^2k^2 - (k', k)^2)}{(w^2k^2 - a^2)}}}{k^2} = Q_{1,2}\\
\end{aligned}
\]

В итоге для отрезка $[t_0; t_1]$ исходного отрезка $S$ получаем:
\begin{equation}
\begin{aligned}
d'_{S'}(t) &= \mathop{lower\_envelope}\{\\
&D(t, t_0), D(t, t_1),\\
&D(t, Q_1t)|_{t \in [Q_1^{-1}t_0; Q_1^{-1}t_1] \cap [0; 1]},\\
&D(t, Q_2t)|_{t \in [Q_2^{-1}t_0; Q_2^{-1}t_1] \cap [0; 1]}\\
&\}
\end{aligned}
\end{equation}

Пусть $\forall i = 1..n: d_S(t) = d^i_S(t), t \in [t_i, t_{i+1}]$,
тогда для всего отрезка получаем:
\begin{equation}
\begin{aligned}
d'_{S'}(t) &= \mathop{lower\_envelope}\{\\
&D^i(t, t_i), D^i(t, t_{i+1}),\\
&D^i(t, Q^i_1t)|_{t \in [{Q^i_1}^{-1}t_i; {Q^i_1}^{-1}t_{i+1}] \cap [0; 1]},\\
&D^i(t, Q^i_2t)|_{t \in [{Q^i_2}^{-1}t_i; {Q^i_2}^{-1}t_{i+1}] \cap [0; 1]}\\
&\}^{i=1..n}
\end{aligned}
\end{equation}

Релаксация превращается в
$d'_S(t) \gets \mathop{lower\_envelope}{M_S(d'_S(t^*), t - t^*)}_{t^* \in T^*}$,
где $T^*$ -- множество точек смены вида функции $d'_S(t)$ (рис. \ref{img:linrelax}).

\drawfigurex{linrelax}{Релаксация кусочно-линейного обновления}{img:linrelax}{width=10cm}

\FloatBarrier

\section{Извлечение района поиска}
После работы алгоритма мы имеем набор ребер с заданным
на них сужением функции кратчайшего расстояния в виде кусочно-линейных
функций.

Рассмотрим каждый треугольник. Расстояние до его внутренних точек
полностью определяется $d_S(t)$ на его сторонах. Значит для нахождения
района поиска необходимо построить район поиска внутри треугольника (рис. \ref{img:triiso}),
а затем построить объединение всех таких этих районов.

\drawfigurex{triiso}{Изолинии $d(P)$ внутри треугольника}{img:triiso}{width=10cm}

Рассмотрим треугольник. Назовем порождающими отрезками отрезки на ребрах,
на которых $d_S(t)$ не меняет свой вид. Считаем концы отрезка не
принадлежащими ему. Назовем переходными точками точки,
разделяющие порождающие отрезки. В итоге треугольник представляется в виде
чередующихся порождающих отрезков и переходных точек. Считаем,
что отрезки изолинии порождаются порождающими отрезками, а
промежуточные точки порождают дуги окружностей.

Рассмотрим подробнее геометрию, порожденную этими объектами для
изолинии со значением $d(t) = d_{iso}$:
\begin{description}
\item[Промежуточная точка] ($P$) \\
Пусть значение функции кратчайшего расстояния в этой точке $d_P$, тогда
$r = (d_{iso} - d_P)/w$, если $r \leq 0$, то никакой геометрии не порождается.
Иначе это окружность радиусом $r$ с центром в $P$ (рис. \ref{img:pointiso}).
\item[Порождающий отрезок] ($S$) \\
Пусть сужение функции кратчайшего расстояния на этот отрезок $d_S(t)$, тогда
если $d_S(0) > d_{iso} \& d_S(1) > d_{iso}$, то никакой геометрии не порождается.
Иначе если $d_S(0) < d_{iso} \& d_S(1) < d_{iso}$, то порождается многоугольник,
состоящая из отрезков общих касательной к окружностям, порожденным предыдущей
и следующей промежуточными точками, и их центрами.
В противном случае берется четырехугольник, состоящий их отрезков касательных
проходящей через $Q = lerp(S, \frac{d_{iso}-d_S(0)}{d_S(1) - d_S(0)})$ к
окружности, порожденной предыдущей промежуточной точкой,
если $d_S(0) < d_{iso}$, следующей иначе, и ее центра (рис. \ref{img:segiso}).
\end{description}

\drawfigure{pointiso}{Изолинии, порожденные промежуточными точками}{img:pointiso}
\drawfigure{segiso}{Изолинии, порожденные порождающими отрезками}{img:segiso}

Далее строится объединение получившихся для треугольника
многоугольников и окружностей. Затем строится пересечение полученной фигуры
с треугольником. В итоге для треугольника получается множество точек,
до которых расстояние не превышает $d_{iso}$.

Последним шагом строится объединение полученных фигур всех треугольников.
Это и будет район поиска.

\FloatBarrier
\section{Точность}

При построении функции кратчайших расстояний $d_S(t)$, была введена
погрешность, а именно, $d_S(t)$ была заменена $d_S(t) - \varepsilon < d'_S(t) < d_S(t)$,
где $\varepsilon$ -- наперед заданная погрешность. Таким образом, при прохождении
одного треугольника вводится погрешность не более $\varepsilon$. То есть
погрешность района поиска -- $N^*_P\varepsilon$, где $N^*_P$ --
максимальное число треугольников, пересекаемых кратчайшим путем
до границ района.

Так как $N^*_P \leq N_T$, где $N_T$ - количество треугольников в триангуляции,
то максимальная возможная погрешность $N_T\varepsilon$.
Но в большинстве случаев кратчайший путь не будет пересекать все треугольники
триангуляции. Более того, можно ожидать, что количество пересеченных
треугольников будет $O(\sqrt{N_T})$. Предположив это, получим оценку погрешности
$O(\sqrt{N_T}\varepsilon)$

%-*-coding: utf-8-*-
\chapter{Тестирование}

\section{Время препроцессирования}

Тестирование проводилось на обычном домашнем компьютере
с процессором Intel Core i3, 1.7 GHz. В качестве тестовых
брались две карты черноморского побережья от г.Анапы до г.Новороссийск
масштаба 1 : 200 000.

Также были выбраны характеристики объекта поиска, примерно соответствующие
характеристикам движения человека. А именно
\begin{description}
\item[Скорость движения по шоссе] -- 5 км/ч;
\item[Скорость движения по грунтовой дороге] -- 4.5 км/ч;
\item[Скорость движения по просеке] -- 4 км/ч;
\item[Скорость движения по лесу] -- 2 км/ч;
\item[Скорость движения по кустарнику] -- 3 км/ч;
\item[Скорость движения по полю] -- 4.5 км/ч;
\item[Скорость движения по городским кварталам] -- 5 км/ч;
\item[Скорость движения по площадным водным объектам] -- 0 км/ч;
\item[Скорость движения по линейным водным объектам] -- 0 км/ч.
\end{description}

Также характеристики объекта поиска учитывают линейные реки.
Была введена штрафная функция, запрещающая пересекать широкие
реки и разрешающая пересекать узкие за 10 мин, но если в
месте пересечения реки ее пересекает дорога, то считается, что
там есть мост и штраф не назначается. Обозначим
$G_S$ -- множество типов местностей ребра $S$, $F$, $T$ --
ребра с которого и на которое осуществляется переход,
$L$, $R$ -- множества ребер, инцидентных той же вершине, что и
$F$, $T$, находящихся слева и справа от них.
Штрафы описывались следующим образом:
\begin{description}
\item["река шириной > 20м" $\in G_F \oplus G_T$] -- $\infty$ мин;
\item["река шириной < 20м" $\in G_F \oplus G_T$] -- 5 мин;
\item["река шириной > 20м" $\displaystyle\in \bigcup_{S\in L}G_S \oplus \bigcup_{S\in R}G_S$]
-- $\infty$ мин, если между берегами реки нет дороги, иначе 0;
\item["река шириной < 20м" $\displaystyle\in \bigcup_{S\in L}G_S \oplus \bigcup_{S\in R}G_S$]
-- 5 мин, если между берегами реки нет дороги, иначе 0;
\item[Иначе] -- 0 мин.
\end{description}

Программа запускалась на двух тестовых картах. На
них произвольно выбирались районы, в которых
далее решалась задача построения района поиска.
Время обработки и характеристики районов приведены в таблицах
\ref{tab:pptime1} и \ref{tab:pptime2}.

\FloatBarrier

\begin{table}[ht]
\centering
\renewcommand{\arraystretch}{1.1}
\captionbox{Время пocтроения функции кратчайших расстояний на карте 1
\label{tab:pptime1}
}{
\begin{tabular*}{1.0\textwidth}{@{\extracolsep{\fill}} |r|r|r|r|}

\hline
Площадь района интереса(км\textsuperscript{2}) & кол-во треугольников & кол-во ребер & время обработки(с) \\
\hline
  18 &   4388 &   19655 &   12 \\
  44 &   9980 &   45134 &   44 \\
 124 &  28840 &  131058 &   88 \\
 374 &  88262 &  404056 &  920 \\
1208 & 260104 & 1192990 & 1403 \\
\hline


\end{tabular*}}
\end{table}

\begin{table}[ht]
\centering
\renewcommand{\arraystretch}{1.1}
\captionbox{Время пocтроения функции кратчайших расстояний на карте 2
\label{tab:pptime2}
}{
\begin{tabular*}{1.0\textwidth}{@{\extracolsep{\fill}} |r|r|r|r|}

\hline
Площадь района интереса(км\textsuperscript{2}) & кол-во треугольников & кол-во ребер & время обработки(с) \\
\hline
  45 &   1170 &   5114 &   4 \\
 127 &   3966 &  17709 &  11 \\
 283 &   9122 &  40803 &  21 \\
 808 &  20844 &  94043 &  57 \\
2268 &  45880 & 207452 & 144 \\
\hline

\end{tabular*}}
\end{table}

\FloatBarrier
\section{Сравнение с другим методом}

Было проведено сравнение предложенного метода с методом,
основанным на растеризации карты в регулярную восьми связную сетку.
Для проведения сравнения, выбирался случайный район интереса (площадью 16, 64, 256,
1024 км\textsuperscript{2}), чей центр считался последней известной
позицией объекта поиска. Далее в этом районе производилась предобработка
данных для обоих методов. В качестве точности предложенного метода
была выбрана точность 4 сек., что примерно соответствует погрешности
5 м. на один треугольник. В качестве шага сетки второго метода были выбраны
те же 5 м., хотя очевидно, что ошибка метризации никогда не позволит
обеспечить такую точность, не увеличив связность сетки. После предобработки
были построены 10 районов поиска. Времена, на которые определялись районы поиска,
выбирались так, чтобы район поиска был полностью внутри района интереса,
для избежания краевых эффектов. В этих целях, максимальное время, на которое
определялся район поиска было выбрано, как половина времени пути до самой
удаленной точки района интереса. Остальные времена выбирались с фиксированным
шагом в сторону уменьшения. Для каждой площади района интереса испытания
проводились десять раз, результаты усреднялись.
Сравнение проводилось автоматически.

В табл. \ref{tab:comparison1}-\ref{tab:comparison4}
приведены результаты сравнительных испытаний этих методов. Как видно из
результатов, предложенный метод работает быстрее метода, основанного
на регулярной сетке, при этом он более точен. В таблицах также приведено
сравнение числа элементов подразбиения, что для предложенного метода
является суммарным число линейных частей кусочно-линейной функции
(для наглядности в скобках указано среднее подразбиение ребра), а для
второго метода -- число ячеек сетки. При ответе на запрос построения района
поиска, показательным является количество многоугольников, которые необходимо
объединить для получения района поиска. Из табл. \ref{tab:comparison4}
видно, что предложенный подход использует меньшее число многоугольников,
что позволяет ему работать в тех случаях, когда другому методу не хватило
оперативной памяти.


%\newcolumntype{R}[1]{D{.}{\cdot}{#1} }
\newcolumntype{R}[1]{>{\raggedleft\arraybackslash}m{#1}}
\newcolumntype{C}[1]{>{\centering\arraybackslash}m{#1}}
\newcolumntype{L}[1]{>{\raggedright\arraybackslash}m{#1}}

\begin{table}[ht]
\centering
\renewcommand{\arraystretch}{1.1}
\captionbox{Сравнение с алгоритмом, использующим регулярную сетку
\label{tab:comparison1}
}{
\begin{tabular*}{1.0\textwidth}{ |C{4cm}|R{2cm}|R{2cm}|R{2cm}|R{2cm}|}
\cline{1-5}
\hfill& \multicolumn{2}{|C{4cm}|}{\textbf{Предложенный алгоритм}} & \multicolumn{2}{C{4cm}|}{\textbf{Регулярная сетка}} \\

\cline{1-5}
Площадь района интереса & \multicolumn{4}{|R{6cm}|}{16 км\textsuperscript{2}} \\
\cline{1-5}
Время препроцессирования (сек)
& \multicolumn{2}{|R{4cm}|}{0.76}
& \multicolumn{2}{R{4cm}|}{1.17}   \\
Кол-во эл. подр.(cр. подр.) / Кол-во ячеек
& \multicolumn{2}{|R{4cm}|}{9156(14.0)}
& \multicolumn{2}{R{4cm}|}{640000} \\
\cline{1-5}
Район поиска на время (мин) & Время построения & Кол-во объектов & Время построения & Кол-во объектов \\
\cline{1-5}
2  & 0.00074 &    8 & 0.0037 &   2905 \\
4  & 0.0021  &  102 & 0.0052 &   9821 \\
6  & 0.0065  &  400 & 0.0071 &  20105 \\
8  & 0.0076  &  640 & 0.013  &  36443 \\
10 & 0.0087  &  768 & 0.018  &  58490 \\
12 & 0.01    &  948 & 0.026  &  85755 \\
14 & 0.012   & 1123 & 0.03   & 116023 \\
16 & 0.012   & 1267 & 0.04   & 148311 \\
18 & 0.014   & 1484 & 0.052  & 177456 \\
20 & 0.014   & 1625 & 0.057  & 205488 \\

\cline{1-5}

\end{tabular*}}
\end{table}

%%%%%%%%%%%%%%%%%%

\begin{table}[ht]
\centering
\renewcommand{\arraystretch}{1.1}
\captionbox{Сравнение с алгоритмом, использующим регулярную сетку
\label{tab:comparison2}
}{
\begin{tabular*}{1.0\textwidth}{ |C{4cm}|R{2cm}|R{2cm}|R{2cm}|R{2cm}|}

\cline{1-5}
Площадь района интереса & \multicolumn{4}{|R{6cm}|}{64 км\textsuperscript{2}} \\
\cline{1-5}
Время препроцессирования (сек)
& \multicolumn{2}{|R{4cm}|}{3.88}
& \multicolumn{2}{R{4cm}|}{5.02}   \\
Кол-во эл. подр.(cр. подр.) / Кол-во ячеек
& \multicolumn{2}{|R{4cm}|}{46920(14.7)}
& \multicolumn{2}{R{4cm}|}{2560000} \\
\cline{1-5}
Район поиска на время (мин) & Время построения & Кол-во объектов & Время построения & Кол-во объектов \\
\cline{1-5}
5  & 0.0056 &   295 & 0.017 &   17333 \\
11 & 0.01   &   836 & 0.03  &   72735 \\
16 & 0.014  &  1326 & 0.055 &  157588 \\
21 & 0.02   &  2149 & 0.087 &  248965 \\
27 & 0.027  &  3134 & 0.11  &  388390 \\
32 & 0.054  &  5901 & 0.18  &  606471 \\
37 & 0.084  &  9677 & 0.21  &  877505 \\
43 & 0.11   & 13389 & 0.3   & 1163230 \\
48 & 0.16   & 19138 & 0.4   & 1418422 \\
53 & 0.19   & 24737 & 0.43  & 1609334 \\
\cline{1-5}

\end{tabular*}}
\end{table}

\begin{table}[ht]
\centering
\renewcommand{\arraystretch}{1.1}
\captionbox{Сравнение с алгоритмом, использующим регулярную сетку
\label{tab:comparison3}
}{
\begin{tabular*}{1.0\textwidth}{ |C{4cm}|R{2cm}|R{2cm}|R{2cm}|R{2cm}|}

\cline{1-5}
Площадь района интереса & \multicolumn{4}{|R{6cm}|}{256 км\textsuperscript{2}} \\
\cline{1-5}
Время препроцессирования (сек)
& \multicolumn{2}{|R{4cm}|}{13.3}
& \multicolumn{2}{R{4cm}|}{26.5}   \\
Кол-во эл. подр.(cр. подр.) / Кол-во ячеек
& \multicolumn{2}{|R{4cm}|}{185326(14.9)}
& \multicolumn{2}{R{4cm}|}{10240000} \\
\cline{1-5}
Район поиска на время (мин) & Время построения & Кол-во объектов & Время построения & Кол-во объектов \\
\cline{1-5}
8  & 0.021 &   622 & 0.059 &   34756 \\
15 & 0.026 &  1257 & 0.085 &  142749 \\
23 & 0.034 &  2400 & 0.13  &  276772 \\
30 & 0.057 &  4914 & 0.18  &  521141 \\
38 & 0.099 &  9905 & 0.32  &  919235 \\
45 & 0.13  & 15375 & 0.46  & 1478305 \\
53 & 0.21  & 25644 & 0.67  & 2196696 \\
60 & 0.24  & 31996 & 0.95  & 3119538 \\
68 & 0.27  & 37986 & 1.1   & 4350680 \\
75 & 0.38  & 51574 & 1.5   & 5655174 \\
\cline{1-5}

\end{tabular*}}
\end{table}

\begin{table}[ht]
\centering
\renewcommand{\arraystretch}{1.1}
\captionbox{Сравнение с алгоритмом, использующим регулярную сетку
\label{tab:comparison4}
}{
\begin{tabular*}{1.0\textwidth}{ |C{4cm}|R{2cm}|R{2cm}|R{2cm}|R{2cm}|}

\cline{1-5}
Площадь района интереса & \multicolumn{4}{|R{6cm}|}{1024 км\textsuperscript{2}} \\
\cline{1-5}
Время препроцессирования (сек)
& \multicolumn{2}{|R{4cm}|}{76}
& \multicolumn{2}{R{4cm}|}{128}   \\
Кол-во эл. подр.(cр. подр.) / Кол-во ячеек
& \multicolumn{2}{|R{4cm}|}{869529(15.3)}
& \multicolumn{2}{R{4cm}|}{40960000} \\
\cline{1-5}
Район поиска на время (мин) & Время построения & Кол-во объектов & Время построения & Кол-во объектов \\
\cline{1-5}
13  & 0.087 & 1069   & 0.22 & 100946 \\
25  & 0.1   & 2844   & 0.28 & 340862 \\
38  & 0.19  & 9907   & 0.46 & 918544 \\
50  & 0.27  & 22380  & 0.65 & 1939039 \\
63  & 0.31  & 33541  & 1.1  & 3492441 \\
75  & 0.46  & 51756  & 1.7  & 5818536 \\
88  & 0.65  & 77954  & 2.5  & 8600438 \\
100 & 0.86  & 106678 & --   & -- \\
113 & 1.2   & 148215 & --   & -- \\
125 & 1.4   & 193374 & --   & -- \\
\cline{1-5}

\end{tabular*}}
\end{table}

\FloatBarrier

\chapter{Применение}

Разработанный метод незаменим при планировании операций
поиска и спасания, особенно в виду того, что текущие
методы учитывают местность и характеристики объекта поиска
лишь эвристически. Для примера приведу отрывок из наставления
по организации и проведению поисково-спасательных работ в
труднодоступной местности\cite{SAR}.

\begin{quotation}
Теоретически район поисков представляет собой круг с центром ПИП и радиусом 10n
(км), где n – число суток незапланированного отсутствия пропавшего. 10 км в сутки
прибавляются в связи с тем, что двигающийся человек проходит в день в среднем такое
расстояние. Он может вообще не двигаться либо проходить по 40 км, однако усредненная
величина для взрослого здорового человека составляет около 10 км.
От круга отсекаются части, ограниченные крупными линейными ориентирами
(железными и шоссейными дорогами, крупными реками). Получившаяся фигура в общем
случае постоянно увеличивается и через 7-10 дней может занимать огромную площадь в
тысячи км\textsuperscript{2} .
Осмотр подобной территории практически невозможен, а наиболее вероятное
местонахождение пропавшего – недалеко от ПИП, поэтому искусственно радиус заменяется
меньшей величиной. Кроме того, РПСР отсекает отдельные части района, основываясь на
объективных соображениях вероятности пребывания пропавшего в том или ином секторе.
\end{quotation}

Применив разработанный метод можно не только построить корректный район поиска
на заданное время, но и производить более правильное
эвристическое уменьшение района поиска.
Например в том же наставлении\cite{SAR} говорится, что в 95\% случаев,
потерявшийся человек находится внутри круга радиусом $\frac{2}{3}$
от исходного, что в терминах моего метода означает, что все скорости
стали на треть меньше, или, в случае отсутствия штрафов, мы рассматриваем
район поиска на время $\frac{2}{3}T$.

Также метод можно применять при поимке сбежавших преступников и поиске
вражеских диверсионных групп. Для последних, с помощь изменения
характеристик объекта поиска, можно добиться более интересных
результатов. Например можно запретить ходить по полям, дорогам,
проходить через населенные пункты.

\FloatBarrier

\startconclusionpage

Как было показано в данной работе, задача определения района поиска
с учетом топографических свойств местности представляет
достаточный практический интерес. Были подробно рассмотрены применяемые
на практике подходы. Предложенный подход отличается не только
корректность определения районов поиска, но и простотой реализации.
Из недостатков данного подхода стоит отметить лишь то, что данный метод
рока не был адаптирован для учета рельефа местности, что позволяло бы
использовать его для планирования поисково-спасательных операций в горной
местности. Но для гористой местности остро стоит вопрос о наличии актуальной
высотной модели, без которой корректность района поиска стоит под сомнением.

В данный момент большой практический интерес представляет ускорение
построения нижних огибающих за счет отказа от точной арифметики в пользу более
быстрых альтернативных техник, таких как, например, adaptive precision
arithmetic \cite{APREC}. Это позволит сократить время на построение, как минимум в
десять раз. Также интересным было бы разработать технику отсечения
заведомо далеких отрезков перед построением нижних огибающих. Что более
важно, реализованный алгоритм очень просто поддается распараллеливанию,
ввиду полной независимости обработки поддеревьев.

Также хотелось бы отметить, что хоть данный метод применялся для
отрезков, он легко обобщается и для других классов объектов. К тому же
данный подход применим и для других классов задач, таких как поиск наиболее
удаленного отрезка (furthest segment problem) и поиск $k$ ближайших соседей
(k-Nearest Neighbor Search). Также данный метод не требует построения
пересечений отрезков для своей работы, в отличии от других рассмотренных
методов.

\FloatBarrier


%\startappendices
%\label{appendix}
%\input{appendix}

\bibliographystyle{sty/utf8gost705u}
\bibliography{thesis}

\end{document}
