%-*-coding: utf-8-*-
\chapter{Существующие методы}

\section{Дискретизация пространства}

Одним из самых распространенных способов решения задач подобного рода является
дискретизация пространства с последующим построением графа из элементов подазбиения.
В основном в таких методах пространство подразбивается регулярной сеткой, реже
адаптивно, структурой схожей с квадродеревом. Точность приближения длины путей
в таких графах на прямую связана характерными размерами элементов подразбиения.
В результате данные методы используют неоправданно большой объем памяти, при
сравнительно малой точности, хотя они и просты в реализации.

\section{Точки Штейнера}
Данный метод базируется на триангуляции исходных данных.
Его работа основана  на двух наблюдениях. Во-первых, в триангуляции исходных
данных каждый треугольник имеет постоянный вес, в результате чего кратчайший
путь внутри треугольника -- прямая. Во-вторых, кратчайший путь подчиняется
закону Снелля из оптики.

Закон Снелля позволяет по углу падения луча на границу раздела двух сред и
скоростям его распространения в этих средах получить угол преломления --
угол, под которым произойдет выход луча из границы раздела сред.
В виду того, что свет движется по кратчайшему пути (принцип Ферма), то этот закон
имеет прямую аналогию в планировании маршрутов.
Рассмотрим ребро триангуляции лежащее на границе районов с разной проходимостью.
Пусть $\theta_1$ -- угол, под которым кратчайший путь входит в это ребро,
$theta_2$ -- угол, под которым он выходит, $v_1, v_2$ -- скорости движения
в районах до и после ребра соответственно. Тогда имеет место равенство:
\begin{equation}
\sin\theta_2 = \sin\theta_1\frac{v_2}{v_1}
\end{equation}

Угол $\theta_c = \arcsin\frac{v_2}{v_1}$ называется критическим углом.
Возможны эффекты так называемого критического отражения от ребра и
критического использования ребра. От ребра происходит критической
отражения маршрута, когда кратчайший путь между двумя точками района
с одной проходимостью выходит из него под критическим углом, а затем
входит обратно, тоже под критическим углом. Это происходит в случае
сильно большей скорости движения в соседнем районе. Критическое
использование ребра происходит в случае прохождения пути между двух районов,
между которыми есть линейный объект с более низкой чем в районах проходимостью.
В этом случае путь под критическим углом входит в это ребро, идет некоторое
расстояние по нему, а затем под критическим углом выходит во второй район.

Закон Снелля позволяет создать более умную схему подразбиения, по которой
на ребра триангуляции, на которых происходит изменение проходимости районов,
добавляются дополнительные точки -- точки Штейнера. Далее триангуляция
используется как граф, в котором внутри треугольника точки Штейнера и вершины
образуют полный граф.

Точность такого метода связана с количеством добавленных точек Штейнера, а
также их распределением по ребру. Во многих работах советуется располагать
больше точек у вершин.

\section{Continuous Dijkstra}


\section{Pathnet}

Алгоритм основан на построении графа, который бы с заданной точностью приближал
расстояния на плоскости.

Из каждой вершины пускаются $k$ лучей, которые далее трассируются в соответствии
с законом Снелля. Пара соседних лучей образует угол. Трассировка идет до тех пор,
пока все лучи, лежащие внутри этого угла ведут себя одинаково. А именно, пока не
произойдет одно из двух событий:
\begin{description}
\item[Лучи-границы конуса прошли через разные ребра одной грани триангуляции]\hfill \\
В этом случае добавляется ребро между исходной вершиной и вершиной, разделившей
эти два луча. Между этими двумя вершинами кратчайший путь обязательно лежит в
конусе. Производится поиск кратчайшего пути, например бинарным поиском, и он
запоминается в ребре, весом ребра назначается вес кратчайшего пути.
\item[Один из лучей-границ вошел в ребро под углом, большим критического]\hfill \\
В этом случае в граф добавляется эта точка -- критическая вершина.
Она соединяется с другими критическими вершинами этого ребра, а также его концами.
Также как и в предыдущем случае, строится кратчайший путь до исходной вершины,
входящий в критическую вершину под критическим углом.
\end{description}

Поиск кратчайшего пути осуществляется вставкой начала и конца пути в граф, а затем
поиском пути с помощью алгоритма дейкстры.

Время работы этого алгоритма $O(kn^3)$, точность $O(\frac{W/\omega}{k\theta_{min}})$,
где $W$ -- максимальный конечный вес, $\omega$ -- минимальный ненулевой вес, $\theta_{min}$ -- минимальный угол
триангуляции
