\chapter{Применение}

Разработанный метод незаменим при планировании операций
поиска и спасания, особенно в виду того, что текущие
методы учитывают местность и характеристики объекта поиска
лишь эвристически. Для примера приведу отрывок из наставления
по организации и проведению поисково-спасательных работ в
труднодоступной местности\cite{SAR}.

\begin{quotation}
Теоретически район поисков представляет собой круг с центром ПИП и радиусом 10n
(км), где n – число суток незапланированного отсутствия пропавшего. 10 км в сутки
прибавляются в связи с тем, что двигающийся человек проходит в день в среднем такое
расстояние. Он может вообще не двигаться либо проходить по 40 км, однако усредненная
величина для взрослого здорового человека составляет около 10 км.
От круга отсекаются части, ограниченные крупными линейными ориентирами
(железными и шоссейными дорогами, крупными реками). Получившаяся фигура в общем
случае постоянно увеличивается и через 7-10 дней может занимать огромную площадь в
тысячи км\textsuperscript{2} .
Осмотр подобной территории практически невозможен, а наиболее вероятное
местонахождение пропавшего – недалеко от ПИП, поэтому искусственно радиус заменяется
меньшей величиной. Кроме того, РПСР отсекает отдельные части района, основываясь на
объективных соображениях вероятности пребывания пропавшего в том или ином секторе.
\end{quotation}

Применив разработанный метод можно не только построить корректный район поиска
на заданное время, но и производить более правильное
эвристическое уменьшение района поиска.
Например в том же наставлении\cite{SAR} говорится, что в 95\% случаев,
потерявшийся человек находится внутри круга радиусом $\frac{2}{3}$
от исходного, что в терминах моего метода означает, что все скорости
стали на треть меньше, или, в случае отсутствия штрафов, мы рассматриваем
район поиска на время $\frac{2}{3}T$.

Также метод можно применять при поимке сбежавших преступников и поиске
вражеских диверсионных групп. Для последних, с помощь изменения
характеристик объекта поиска, можно добиться более интересных
результатов. Например можно запретить ходить по полям, дорогам,
проходить через населенные пункты.

\FloatBarrier
