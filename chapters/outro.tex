\startconclusionpage

Как было показано в данной работе, задача определения района поиска
с учетом топографических свойств местности представляет
достаточный практический интерес. Были подробно рассмотрены применяемые
на практике подходы. Предложенный подход отличается не только
корректность определения районов поиска, но и простотой реализации.
Из недостатков данного подхода стоит отметить лишь то, что данный метод
рока не был адаптирован для учета рельефа местности, что позволяло бы
использовать его для планирования поисково-спасательных операций в горной
местности. Но для гористой местности остро стоит вопрос о наличии актуальной
высотной модели, без которой корректность района поиска стоит под сомнением.

В данный момент большой практический интерес представляет ускорение
построения нижних огибающих за счет отказа от точной арифметики в пользу более
быстрых альтернативных техник, таких как, например, adaptive precision
arithmetic \cite{APREC}. Это позволит сократить время на построение, как минимум в
десять раз. Также интересным было бы разработать технику отсечения
заведомо далеких отрезков перед построением нижних огибающих. Что более
важно, реализованный алгоритм очень просто поддается распараллеливанию,
ввиду полной независимости обработки поддеревьев.

Также хотелось бы отметить, что хоть данный метод применялся для
отрезков, он легко обобщается и для других классов объектов. К тому же
данный подход применим и для других классов задач, таких как поиск наиболее
удаленного отрезка (furthest segment problem) и поиск $k$ ближайших соседей
(k-Nearest Neighbor Search). Также данный метод не требует построения
пересечений отрезков для своей работы, в отличии от других рассмотренных
методов.

\FloatBarrier
