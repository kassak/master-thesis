\startconclusionpage

Как было показано в данной работе, задача определения района поиска
с учетом топографических свойств местности представляет
достаточный практический интерес. Были подробно рассмотрены применяемые
на практике подходы. Предложенный подход отличается не только
корректность определения районов поиска, но и простотой реализации.
Из недостатков данного подхода стоит отметить лишь то, что данный метод
пока не был адаптирован для учета рельефа местности, что позволяло бы
использовать его для планирования поисково-спасательных операций в горной
местности. Но для гористой местности остро стоит вопрос о наличии актуальной
высотной модели, без которой корректность попыток построения
района поиска стоит под сомнением.

В данный момент большой практический интерес представляет ускорение
построения нижних огибающих за счет отказа от точной арифметики в пользу более
быстрых альтернативных техник, таких как, например, adaptive precision
arithmetic \cite{APREC}. Это позволит сократить время на построение, как минимум в
десять раз. Также интересным вопросом является возможность использования
функций другого вида для представления сужения поля на ребра, например
кусочно-квадратичные функции.

Также хотелось бы отметить, что, хоть данный метод применялся для
получения нижней оценки на кратчайшее расстояние, его можно применить
для получения верхней оценки на кратчайшие расстояния, а также для получения
самих путей, полученных алгоритмом.

\FloatBarrier
