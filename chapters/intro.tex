% -*-coding: utf-8-*-
\startprefacepage

При при проведении поисково-спасательных операций, одним
из основных факторов, определяющих ее успех, является время.
Поэтому правильное определение района поиска может стать залогом
успеха операции.

\begin{description}
\item[Исходный район] -- область, где в последний раз был замечен объект поиска.
\item[Достижимый район на время $t$] -- множество точек, достижимых
из точек исходного района за время $t$.
\item[Район поиска] -- область в которой осуществляется поиск.
\item[Корректный район поиска на время $t$] -- район поиска, полностью
содержащий в себе достижимый район на время $t$
\end{description}

Задача заключается в определении корректного района поиска на заданный
момент времени, по известным картографическим данным, модели
объекта поиска и исходному району, отличающийся от достижимого района
не более чем на заданную величину.

На сегодняшний, при планировании поиска и спасания производится
очень грубая оценка района поиска, что ведет к бессмысленной трате
времени и ресурсов на поиск в местах, где объкта поиска точно нет.

Очевидно, что для этой задачи есть необходимость получать
районы поиска на разные моменты времени при одних и тех же
входных данных. Например просматривать динамику расширения района
поиска с течением времени. Это означает что запросы
районов поиска будут массовыми.

Массовая задача в \cite{PrSh} определена следующим образом. Существует
фиксированный набор входных данных $S$. Требуется вычислить массовый
запрос $Q$, то есть ответить на некоторый поставленный вопрос для каждого
запроса из $Q$. Иногда такие задачи решаются в два этапа: предобработка
(pre-processing) и вычисление запросов на некоторой структуре данных,
формирующейся на этапе предобработки и ускоряющих вычисления, что позволяет
сократить суммарное время по сравнению с последовательным решением
исходной задачи для каждого запроса.

В данной работе описан новый метод массового построения районов
поиска на заданные моменты времени, которые с одной стороны являются корректными,
а с другой стороны отличаются от достижимых на эти моменты
времени районы не более чем на заданную величину.

\begin{comment}
Дано:
\begin{itemize}
\item карта -- множество многоугольников и полилиний с указанием типа местности;
\item модель объекта поиска (МОП) -- предположение об ОП, а именно максимальные скорости
ОП для всех типов местности;
\item исходный район (ИР) -- область пространства, в которой мог находиться ОП
в момент времени $T_0$.
\end{itemize}

Задача нахождения $\epsilon-минимального$ корректного РП состоит в нахождении
полигонального приближения ДР.
Задача нахождения ДР -- задача поиска кратчайших расстояний до всех точек от
исходной (One-Source Shortest Path).
Для случая полигональных препядствий с весами $[0; \inf]$ кратчайшее расстояние
не может быть найдено аналитически. В таком случае ищут $\epsilon$-оптимальные пути.
\todo{eps-opt}
Существующие методы позволяют находить $\epsilon$-оптимальные пути, но их нельзя
явно применить для решения данной задачи. Допустим алгоритм нашел путь -- $r(s, t)$,
но он не кратчайший -- есть кратчайший путь $r^*(s, t)$. Так как $|r|$ > $|r^*|$, то
если рассмотреть множество точек $M$ на расстоянии $R$, полученном данным алгоритмом,
то $M \in M^*$, где $M^*$ -- район на расстоянии $R$. А значит, полученный район не
будет коррекным. Значит требуется получить нижнюю оценку на кратчайшее растояние.
\end{comment}
