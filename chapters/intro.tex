% -*-coding: utf-8-*-
\startprefacepage

При при проведении поисково-спасательных операций, одним
из основных факторов, определяющих ее успех, является время.
Поэтому правильное определение района поиска может стать залогом
успеха операции.

\begin{description}
\item[Исходный район] -- область, где в последний раз был замечен объект поиска.
\item[Достижимый район на время $t$] -- множество точек, достижимых
из точек исходного района за время $t$.
\item[Район поиска] -- область в которой осуществляется поиск.
\item[Корректный район поиска на время $t$] -- район поиска, полностью
содержащий в себе достижимый район на время $t$
\end{description}

Задача заключается в определении корректного района поиска на заданный
момент времени, по известным картографическим данным, модели
объекта поиска и исходному району, отличающийся от достижимого района
не более чем на заданную величину.

\drawfigure{example}{Пример района поиска}{img:ex}

На сегодняшний, при планировании поиска и спасания производится
очень грубая оценка района поиска, что ведет к бессмысленной трате
времени и ресурсов на поиск в местах, где объкта поиска точно нет \cite{SAR}.

Очевидно, что для этой задачи есть необходимость получать
районы поиска на разные моменты времени при одних и тех же
входных данных. Например просматривать динамику расширения района
поиска с течением времени. Это означает что запросы
районов поиска будут массовыми.

Массовая задача в \cite{PrSh} определена следующим образом. Существует
фиксированный набор входных данных $S$. Требуется вычислить массовый
запрос $Q$, то есть ответить на некоторый поставленный вопрос для каждого
запроса из $Q$. Иногда такие задачи решаются в два этапа: предобработка
(pre-processing) и вычисление запросов на некоторой структуре данных,
формирующейся на этапе предобработки и ускоряющих вычисления, что позволяет
сократить суммарное время по сравнению с последовательным решением
исходной задачи для каждого запроса.

В данной работе описан новый метод массового построения районов
поиска на заданные моменты времени, которые с одной стороны являются корректными,
а с другой стороны отличаются от достижимых на эти моменты
времени районы не более чем на заданную величину.

\FloatBarrier
