% -*-coding: utf-8-*-
\startprefacepage

При при проведении поисково-спасательных операций, одним
из основных факторов, определяющих ее успех, является время.
Поэтому правильное определение района поиска может стать залогом
успеха операции.

На сегодняшний, при планировании поиска и спасания производится
очень грубая оценка района поиска, что ведет к бессмысленной трате
времени и ресурсов на поиск в местах, где объкта поиска точно нет.

--------------------------------

В данной работе рассмотрено построение $\epsilon-минимального$ корректного РП
на момент времени $T$ для случая, когда нам известна область (или точка),
в которой мог находиться объект поиска (ОП) в момент времени $T_0$.

Достижимый район (ДР) -- область пространства, достижимая из ИР с учетом МОП.
Район поиска -- полигональная область пространства, в которой осуществляется поиск.
РП -- корректный, если он содержит в себе ДР.
\todo{$\epsilon-минимальность$}

Дано:
\begin{itemize}
\item карта -- множество многоугольников и полилиний с указанием типа местности;
\item модель объекта поиска (МОП) -- предположение об ОП, а именно максимальные скорости
ОП для всех типов местности;
\item исходный район (ИР) -- область пространства, в которой мог находиться ОП
в момент времени $T_0$.
\end{itemize}

Задача нахождения $\epsilon-минимального$ корректного РП состоит в нахождении
полигонального приближения ДР.
Задача нахождения ДР -- задача поиска кратчайших расстояний до всех точек от
исходной (One-Source Shortest Path).
Для случая полигональных препядствий с весами $[0; \inf]$ кратчайшее расстояние
не может быть найдено аналитически. В таком случае ищут $\epsilon$-оптимальные пути.
\todo{eps-opt}
Существующие методы позволяют находить $\epsilon$-оптимальные пути, но их нельзя
явно применить для решения данной задачи. Допустим алгоритм нашел путь -- $r(s, t)$,
но он не кратчайший -- есть кратчайший путь $r^*(s, t)$. Так как $|r|$ > $|r^*|$, то
если рассмотреть множество точек $M$ на расстоянии $R$, полученном данным алгоритмом,
то $M \in M^*$, где $M^*$ -- район на расстоянии $R$. А значит, полученный район не
будет коррекным. Значит требуется получить нижнюю оценку на кратчайшее растояние.
