%-*-coding: utf-8-*-
\chapter{Тестирование}

Тестирование проводилось на обычном домашнем компьютере
с процессором Intel Core i3, 1.7 GHz. В качестве тестовых
брались две карты черноморского побережья от г.Анапы до г.Новороссийск
масштаба 1 : 200 000.

Также были выбраны характеристики объекта поиска, примерно соответствующие
характеристикам движения человека. А именно
\begin{description}
\item[Скорость движения по шоссе] -- 5 км/ч;
\item[Скорость движения по грунтовой дороге] -- 4.5 км/ч;
\item[Скорость движения по просеке] -- 4 км/ч;
\item[Скорость движения по лесу] -- 2 км/ч;
\item[Скорость движения по кустарнику] -- 3 км/ч;
\item[Скорость движения по полю] -- 4.5 км/ч;
\item[Скорость движения по городским кварталам] -- 5 км/ч;
\item[Скорость движения по площадным водным объектам] -- 0 км/ч;
\item[Скорость движения по линейным водным объектам] -- 0 км/ч.
\end{description}

Также характеристики объекта поиска учитывают линейные реки.
Была введена штрафная функция, запрещающая пересекать широкие
реки и разрешающая пересекать узкие за 10 мин, но если в
месте пересечения реки ее пересекает дорога, то считается, что
там есть мост и штраф не назначается. Обозначим
$G_S$ -- множество типов местностей ребра $S$, $F$, $T$ --
ребра с которого и на которое осуществляется переход,
$L$, $R$ -- множества ребер, инцидентных той же вершине, что и
$F$, $T$, находящихся слева и справа от них.
Штрафы описывались следующим образом:
\begin{description}
\item["река шириной > 20м" $\in G_F \oplus G_T$] -- $\infty$ мин;
\item["река шириной < 20м" $\in G_F \oplus G_T$] -- 5 мин;
\item["река шириной > 20м" $\displaystyle\in \bigcup_{S\in L}G_S \oplus \bigcup_{S\in R}G_S$]
-- $\infty$ мин, если между берегами реки нет дороги, иначе 0;
\item["река шириной < 20м" $\displaystyle\in \bigcup_{S\in L}G_S \oplus \bigcup_{S\in R}G_S$]
-- 5 мин, если между берегами реки нет дороги, иначе 0;
\item[Иначе] -- 0 мин.
\end{description}

Программа запускалась на двух тестовых картах. На
них произвольно выбирались районы, в которых
далее решалась задача построения района поиска.
Время обработки и характеристики районов приведены в таблицах
\ref{tab:pptime1} и \ref{tab:pptime2}.

\FloatBarrier

\begin{table}[ht]
\centering
\renewcommand{\arraystretch}{1.1}
\captionbox{Время пocтроения функции кратчайших расстояний на карте 1
\label{tab:pptime1}
}{
\begin{tabular*}{1.0\textwidth}{@{\extracolsep{\fill}} |r|r|r|r|}

\hline
Площадь района интереса(км\textsuperscript{2}) & кол-во треугольников & кол-во ребер & время обработки(с) \\
\hline
  18 &   4388 &   19655 &   12 \\
  44 &   9980 &   45134 &   44 \\
 124 &  28840 &  131058 &   88 \\
 374 &  88262 &  404056 &  920 \\
1208 & 260104 & 1192990 & 1403 \\
\hline


\end{tabular*}}
\end{table}

\begin{table}[ht]
\centering
\renewcommand{\arraystretch}{1.1}
\captionbox{Время пocтроения функции кратчайших расстояний на карте 2
\label{tab:pptime2}
}{
\begin{tabular*}{1.0\textwidth}{@{\extracolsep{\fill}} |r|r|r|r|}

\hline
Площадь района интереса(км\textsuperscript{2}) & кол-во треугольников & кол-во ребер & время обработки(с) \\
\hline
  45 &   1170 &   5114 &   4 \\
 127 &   3966 &  17709 &  11 \\
 283 &   9122 &  40803 &  21 \\
 808 &  20844 &  94043 &  57 \\
2268 &  45880 & 207452 & 144 \\
\hline

\end{tabular*}}
\end{table}

%\newcolumntype{R}[1]{D{.}{\cdot}{#1} }
\newcolumntype{R}[1]{>{\raggedleft\arraybackslash}m{#1}}
\newcolumntype{C}[1]{>{\centering\arraybackslash}m{#1}}
\newcolumntype{L}[1]{>{\raggedright\arraybackslash}m{#1}}

\begin{table}[ht]
\centering
\renewcommand{\arraystretch}{1.1}
\captionbox{Сравнение с алгоритмом, использующим регулярную сетку
\label{tab:comparison1}
}{
\begin{tabular*}{1.0\textwidth}{ |C{4cm}|R{2cm}|R{2cm}|R{2cm}|R{2cm}|}
\cline{1-5}
\hfill& \multicolumn{2}{|C{4cm}|}{\textbf{Предложенный алгоритм}} & \multicolumn{2}{C{4cm}|}{\textbf{Регулярная сетка}} \\

\cline{1-5}
Площадь района интереса & \multicolumn{4}{|R{6cm}|}{16 км\textsuperscript{2}} \\
\cline{1-5}
Время препроцессирования (сек)      & \multicolumn{2}{|R{4cm}|}{0.76} & \multicolumn{2}{R{4cm}|}{1.17}   \\
Среднее подразбиение / Кол-во ячеек & \multicolumn{2}{|R{4cm}|}{14.0} & \multicolumn{2}{R{4cm}|}{640000} \\
\cline{1-5}
Район поиска на время (сек) & Время построения & Кол-во объектов & Время построения & Кол-во объектов \\
\cline{1-5}
160.352m & 0.00074 &    8 & 0.0037 &   2905 \\
320.703m & 0.0021  &  102 & 0.0052 &   9821 \\
481.055m & 0.0065  &  400 & 0.0071 &  20105 \\
641.406m & 0.0076  &  640 & 0.013  &  36443 \\
801.758m & 0.0087  &  768 & 0.018  &  58490 \\
962.109m & 0.01    &  948 & 0.026  &  85755 \\
1122.46m & 0.012   & 1123 & 0.03   & 116023 \\
1282.81m & 0.012   & 1267 & 0.04   & 148311 \\
1443.16m & 0.014   & 1484 & 0.052  & 177456 \\
1603.52m & 0.014   & 1625 & 0.057  & 205488 \\

\cline{1-5}

\end{tabular*}}
\end{table}

%%%%%%%%%%%%%%%%%%

\begin{table}[ht]
\centering
\renewcommand{\arraystretch}{1.1}
\captionbox{Сравнение с алгоритмом, использующим регулярную сетку
\label{tab:comparison2}
}{
\begin{tabular*}{1.0\textwidth}{ |C{4cm}|R{2cm}|R{2cm}|R{2cm}|R{2cm}|}

\cline{1-5}
Площадь района интереса & \multicolumn{4}{|R{6cm}|}{64 км\textsuperscript{2}} \\
\cline{1-5}
Время препроцессирования (сек)      & \multicolumn{2}{|R{4cm}|}{3.88} & \multicolumn{2}{R{4cm}|}{5.02}   \\
Среднее подразбиение / Кол-во ячеек & \multicolumn{2}{|R{4cm}|}{14.7} & \multicolumn{2}{R{4cm}|}{2560000} \\
\cline{1-5}
Район поиска на время (сек) & Время построения & Кол-во объектов & Время построения & Кол-во объектов \\
\cline{1-5}
443.919m & 0.0056 &   295 & 0.017 &   17333 \\
887.839m & 0.01   &   836 & 0.03  &   72735 \\
1331.76m & 0.014  &  1326 & 0.055 &  157588 \\
1775.68m & 0.02   &  2149 & 0.087 &  248965 \\
2219.60m & 0.027  &  3134 & 0.11  &  388390 \\
2663.52m & 0.054  &  5901 & 0.18  &  606471 \\
3107.44m & 0.084  &  9677 & 0.21  &  877505 \\
3551.35m & 0.11   & 13389 & 0.3   & 1163230 \\
3995.27m & 0.16   & 19138 & 0.4   & 1418422 \\
4439.19m & 0.19   & 24737 & 0.43  & 1609334 \\
\cline{1-5}

\end{tabular*}}
\end{table}

\begin{table}[ht]
\centering
\renewcommand{\arraystretch}{1.1}
\captionbox{Сравнение с алгоритмом, использующим регулярную сетку
\label{tab:comparison3}
}{
\begin{tabular*}{1.0\textwidth}{ |C{4cm}|R{2cm}|R{2cm}|R{2cm}|R{2cm}|}

\cline{1-5}
Площадь района интереса & \multicolumn{4}{|R{6cm}|}{256 км\textsuperscript{2}} \\
\cline{1-5}
Время препроцессирования (сек)      & \multicolumn{2}{|R{4cm}|}{13.3} & \multicolumn{2}{R{4cm}|}{26.5}   \\
Среднее подразбиение / Кол-во ячеек & \multicolumn{2}{|R{4cm}|}{14.9} & \multicolumn{2}{R{4cm}|}{10240000} \\
\cline{1-5}
Район поиска на время (сек) & Время построения & Кол-во объектов & Время построения & Кол-во объектов \\
\cline{1-5}
626.984m & 0.021 &   622 & 0.059 &   34756 \\
1253.97m & 0.026 &  1257 & 0.085 &  142749 \\
1880.95m & 0.034 &  2400 & 0.13  &  276772 \\
2507.94m & 0.057 &  4914 & 0.18  &  521141 \\
3134.92m & 0.099 &  9905 & 0.32  &  919235 \\
3761.90m & 0.13  & 15375 & 0.46  & 1478305 \\
4388.89m & 0.21  & 25644 & 0.67  & 2196696 \\
5015.87m & 0.24  & 31996 & 0.95  & 3119538 \\
5642.86m & 0.27  & 37986 & 1.1   & 4350680 \\
6269.84m & 0.38  & 51574 & 1.5   & 5655174 \\
\cline{1-5}

\end{tabular*}}
\end{table}

\begin{table}[ht]
\centering
\renewcommand{\arraystretch}{1.1}
\captionbox{Сравнение с алгоритмом, использующим регулярную сетку
\label{tab:comparison4}
}{
\begin{tabular*}{1.0\textwidth}{ |C{4cm}|R{2cm}|R{2cm}|R{2cm}|R{2cm}|}

\cline{1-5}
Площадь района интереса & \multicolumn{4}{|R{6cm}|}{1024 км\textsuperscript{2}} \\
\cline{1-5}
Время препроцессирования (сек)      & \multicolumn{2}{|R{4cm}|}{76} & \multicolumn{2}{R{4cm}|}{128}   \\
Среднее подразбиение / Кол-во ячеек & \multicolumn{2}{|R{4cm}|}{15.3} & \multicolumn{2}{R{4cm}|}{40960000} \\
\cline{1-5}
Район поиска на время (сек) & Время построения & Кол-во объектов & Время построения & Кол-во объектов \\
\cline{1-5}
1044.67m & 0.087 & 1069   & 0.22 & 100946 \\
2089.33m & 0.1   & 2844   & 0.28 & 340862 \\
3134.00m & 0.19  & 9907   & 0.46 & 918544 \\
4178.67m & 0.27  & 22380  & 0.65 & 1939039 \\
5223.34m & 0.31  & 33541  & 1.1  & 3492441 \\
6268.00m & 0.46  & 51756  & 1.7  & 5818536 \\
7312.67m & 0.65  & 77954  & 2.5  & 8600438 \\
8357.34m & 0.86  & 106678 & --   & -- \\
9402.00m & 1.2   & 148215 & --   & -- \\
10446.7m & 1.4   & 193374 & --   & -- \\
\cline{1-5}

\end{tabular*}}
\end{table}

\FloatBarrier
