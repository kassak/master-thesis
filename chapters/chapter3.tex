%-*-coding: utf-8-*-
\chapter{Тестирование}

Тестирование проводилось на обычном домашнем компьютере
с процессором Intel Core i3, 1.7 GHz. В качестве тестовых
брались две карты черноморского побережья от г.Анапы до г.Новороссийск
масштаба 1 : 200 000.

Также были выбраны характеристики объекта поиска, примерно соответствующие
характеристикам движения человека. А именно
\begin{description}
\item[Скорость движения по шоссе] -- 5 км/ч;
\item[Скорость движения по грунтовой дороге] -- 4.5 км/ч;
\item[Скорость движения по просеке] -- 4 км/ч;
\item[Скорость движения по лесу] -- 2 км/ч;
\item[Скорость движения по кустарнику] -- 3 км/ч;
\item[Скорость движения по полю] -- 4.5 км/ч;
\item[Скорость движения по городским кварталам] -- 5 км/ч;
\item[Скорость движения по площадным водным объектам] -- 0 км/ч;
\item[Скорость движения по линейным водным объектам] -- 0 км/ч.
\end{description}

Также характеристики объекта поиска учитывают линейные реки.
Была введена штрафная функция, запрещающая пересекать широкие
реки и разрешающая пересекать узкие за 10 мин, но если в
месте пересечения реки ее пересекает дорога, то считается, что
там есть мост и штраф не назначается. Обозначим
$G_S$ -- множество типов местностей ребра $S$, $F$, $T$ --
ребра с которого и на которое осуществляется переход,
$L$, $R$ -- множества ребер, инцидентных той же вершине, что и
$F$, $T$, находящихся слева и справа от них.
Штрафы описывались следующим образом:
\begin{description}
\item["река шириной > 20м" $\in G_F \oplus G_T$] -- $\infty$ мин;
\item["река шириной < 20м" $\in G_F \oplus G_T$] -- 5 мин;
\item["река шириной > 20м" $\displaystyle\in \bigcup_{S\in L}G_S \oplus \bigcup_{S\in R}G_S$]
-- $\infty$ мин, если между берегами реки нет дороги, иначе 0;
\item["река шириной < 20м" $\displaystyle\in \bigcup_{S\in L}G_S \oplus \bigcup_{S\in R}G_S$]
-- 5 мин, если между берегами реки нет дороги, иначе 0;
\item[Иначе] -- 0 мин.
\end{description}

Программа запускалась на двух тестовых картах. На
них произвольно выбирались районы, в которых
далее решалась задача построения района поиска.
Время обработки и характеристики районов приведены в таблицах
\ref{tab:pptime1} и \ref{tab:pptime2}.

\FloatBarrier

\begin{table}[ht]
\centering
\renewcommand{\arraystretch}{1.1}
\captionbox{Время пocтроения функции кратчайших расстояний на карте 1
\label{tab:pptime1}
}{
\begin{tabular*}{1.0\textwidth}{@{\extracolsep{\fill}} |r|r|r|r|}

\hline
Площадь района интереса(км\textsuperscript{2}) & кол-во треугольников & кол-во ребер & время обработки(с) \\
\hline
  18 &   4388 &   19655 &   12 \\
  44 &   9980 &   45134 &   44 \\
 124 &  28840 &  131058 &   88 \\
 374 &  88262 &  404056 &  920 \\
1208 & 260104 & 1192990 & 1403 \\
\hline


\end{tabular*}}
\end{table}

\begin{table}[ht]
\centering
\renewcommand{\arraystretch}{1.1}
\captionbox{Время пocтроения функции кратчайших расстояний на карте 2
\label{tab:pptime2}
}{
\begin{tabular*}{1.0\textwidth}{@{\extracolsep{\fill}} |r|r|r|r|}

\hline
Площадь района интереса(км\textsuperscript{2}) & кол-во треугольников & кол-во ребер & время обработки(с) \\
\hline
  45 &   1170 &   5114 &   4 \\
 127 &   3966 &  17709 &  11 \\
 283 &   9122 &  40803 &  21 \\
 808 &  20844 &  94043 &  57 \\
2268 &  45880 & 207452 & 144 \\
\hline

\end{tabular*}}
\end{table}

\FloatBarrier
